\documentclass[runningheads]{llncs}

\usepackage{graphicx}


\begin{document}
%
\title{Behavioral Biometric Classifier on 8051 Microcontroller\thanks{Supported by CSEN 702: Microprocessors Course}}

\author{Abdullah ElKady \and Mohamed Ibrahim \and Amr Kayid \and Ahmed Shawky \and Ahmed Sabek}


\institute{German University in Cairo, Egypt \\
\email{\{abdullah.kady, mohamed.meeza, amr.kayid, ahmed.shawky, ahmed.sabek\}@guc.edu.eg}}

\maketitle             


\begin{abstract}
A simple behavioral biometric classifier on 8051 Microcontroller that can identify users based on their keystroke dynamics profiles implemented using C language on Keil C51 development tool.

\keywords{Behavioral Biometric  \and Classifier \and Microcontroller \and Dwell time \and Flight time.}
\end{abstract}

%
%
%

\section{Introduction}
\subsection{behavioral biometric}
In this project we used the behavioral biometric of Keystroke Dynamics to identify individual types characters on a keyboard or keypad. The keystroke rhythms of a user are measured to develop a unique biometric profile of the user's typing pattern for future authentication, the project idea is similar to the one that was used by Coursera~\cite{ref_article1} for verifying learners before submitting graded quizzes and assignments. \\

Keystroke dynamics can therefore be described as a software-based algorithm that measures both dwell and flight time to authenticate identity. where dwell time is the duration that a key is pressed, while flight time is the duration between keystrokes.

\subsubsection{Project Idea:} Only two levels of
headings should be numbered. Lower level headings remain unnumbered;
they are formatted as run-in headings.

\paragraph{Sample Heading (Fourth Level)}
The contribution should contain no more than four levels of
headings. Table~\ref{tab1} gives a summary of all heading levels.

\begin{table}
\caption{Table captions should be placed above the
tables.}\label{tab1}
\begin{tabular}{|l|l|l|}
\hline
Heading level &  Example & Font size and style\\
\hline
Title (centered) &  {\Large\bfseries Lecture Notes} & 14 point, bold\\
1st-level heading &  {\large\bfseries 1 Introduction} & 12 point, bold\\
2nd-level heading & {\bfseries 2.1 Printing Area} & 10 point, bold\\
3rd-level heading & {\bfseries Run-in Heading in Bold.} Text follows & 10 point, bold\\
4th-level heading & {\itshape Lowest Level Heading.} Text follows & 10 point, italic\\
\hline
\end{tabular}
\end{table}


\noindent Displayed equations are centered and set on a separate
line.
\begin{equation}
x + y = z
\end{equation}
Please try to avoid rasterized images for line-art diagrams and
schemas. Whenever possible, use vector graphics instead (see
Fig.~\ref{fig1}).

\begin{figure}
\includegraphics[width=\textwidth]{fig1.eps}
\caption{A figure caption is always placed below the illustration.
Please note that short captions are centered, while long ones are
justified by the macro package automatically.} \label{fig1}
\end{figure}

\begin{theorem}
This is a sample theorem. The run-in heading is set in bold, while
the following text appears in italics. Definitions, lemmas,
propositions, and corollaries are styled the same way.
\end{theorem}
%
% the environments 'definition', 'lemma', 'proposition', 'corollary',
% 'remark', and 'example' are defined in the LLNCS documentclass as well.
%
\begin{proof}
Proofs, examples, and remarks have the initial word in italics,
while the following text appears in normal font.
\end{proof}
For citations of references, we prefer the use of square brackets
and consecutive numbers. Citations using labels or the author/year
convention are also acceptable. The following bibliography provides
a sample reference list with entries for journal
articles~\cite{ref_article1}, an LNCS chapter~\cite{ref_lncs1}, a
book~\cite{ref_book1}, proceedings without editors~\cite{ref_proc1},
and a homepage~\cite{ref_url1}. Multiple citations are grouped
\cite{ref_article1,ref_lncs1,ref_book1},
\cite{ref_article1,ref_book1,ref_proc1,ref_url1}.
%
% ---- Bibliography ----
%
% BibTeX users should specify bibliography style 'splncs04'.
% References will then be sorted and formatted in the correct style.
%
% \bibliographystyle{splncs04}
% \bibliography{mybibliography}
%
\begin{thebibliography}{8}
\bibitem{ref_article1}
Ng, A.: Ubiquity symposium: MOOCs and technology to advance learning and learning research: offering verified credentials in massive open online courses. ACM

\bibitem{ref_lncs1}
Author, F., Author, S.: Title of a proceedings paper. In: Editor,
F., Editor, S. (eds.) CONFERENCE 2016, LNCS, vol. 9999, pp. 1--13.
Springer, Heidelberg (2016). \doi{10.10007/1234567890}

\bibitem{ref_book1}
Author, F., Author, S., Author, T.: Book title. 2nd edn. Publisher,
Location (1999)

\bibitem{ref_proc1}
Author, A.-B.: Contribution title. In: 9th International Proceedings
on Proceedings, pp. 1--2. Publisher, Location (2010)

\bibitem{ref_url1}
LNCS Homepage, \url{http://www.springer.com/lncs}. Last accessed 4
Oct 2017
\end{thebibliography}
\end{document}
